\documentclass[10pt, a4paper,twoside,american]{article}
\usepackage{geometry}
\geometry{verbose,a4paper,tmargin=2cm,bmargin=2cm,lmargin=1.375in,rmargin=1.375in,headheight=7mm,headsep=6mm,footskip=10mm}

%%%%%%%%%%%%%%%%%%%%%%%%%%%%%%%%%%%%%%%%%%%%%%%%%%%%%%%%%%%%%%%%%%%%%%%%%%%%%
%% title and short title
\newcommand{\maintitle}{INM6 retreat 2016}
\newcommand{\shorttitle}{Sessions}
%%%%%%%%%%%%%%%%%%%%%%%%%%%%%%%%%%%%%%%%%%%%%%%%%%%%%%%%%%%%%%%%%%%%%%%%%%%%%
\usepackage{calc}
\usepackage{amsmath}
\usepackage{amssymb}
\usepackage{graphicx}
\usepackage{calc}
\usepackage{color} 
\usepackage{url}
\usepackage[labelfont=bf,labelsep=period,justification=raggedright]{caption}
\usepackage{natbib}
\usepackage{float}
\usepackage{prettyref}
\usepackage[%%
breaklinks=true,
colorlinks=true,
linkcolor=blue,citecolor=blue,urlcolor=blue,filecolor=blue,
pdffitwindow,
%backref,
bookmarks=true,
bookmarksopen=true,
bookmarksnumbered=true,
pdftitle={\maintitle},
pdfauthor={},
pdfsubject={},
pdfkeywords={}
]
{hyperref}
\usepackage{setspace}
\usepackage{lscape}
\usepackage{bm}
\usepackage{enumitem}
\setenumerate[1]{label=\large\bf\arabic*.}
%\setenumerate[1]{label=\thesection.\arabic*}
%\setenumerate[2]{label*=\arabic*.}
%%%%%%%%%%%%%%%%%%%%%%%%%%%%%%%%%%%%%%%%%%%%%%%%%%%%%%%%%%%%%%%%%%%%%%%%%%%%%
%% layout stuff

\pagestyle{empty}  % no header of footer

%% headers, footers
\usepackage{fancyhdr}
\pagestyle{fancy}
\renewcommand{\headrulewidth}{0pt}
\fancyhead{}
\fancyhead[L]{\maintitle: \shorttitle}
\fancyhead[R]{\thepage}
\fancyfoot[C]{}

\renewcommand{\familydefault}{cmss} %% sans serif fonts

\newcommand{\alert}[1]{\textcolor{red}{\bf #1}}

%%%%%%%%%%%%%%%%%%%%%%%%%%%%%%%%%%%%%%%%%%%%%%%%%%%%%%%%%%%%%%%%%%%%%%%%
%%%%%%%%%%%%%%%%%%%%%%%%%%%%%%%%%%%%%%%%%%%%%%%%%%%%%%%%%%%%%%%%%%%%%%%%
\begin{document}
%%%%%%%%%%%%%%%%%%%%%%%%%%%%%%%%%%%%%%%%%%%%%%%%%%%%%%%%%%%%%%%%%%%%%%%%
%%%%%%%%%%%%%%%%%%%%%%%%%%%%%%%%%%%%%%%%%%%%%%%%%%%%%%%%%%%%%%%%%%%%%%%%
%% title page
\pdfbookmark[1]{Title}{TitlePage}
\noindent
\thispagestyle{empty}
\parbox{\textwidth}{\centering
 {\huge\bf \maintitle: \shorttitle}}
%
\vspace*{1cm}
%%%%%%%%%%%%%%%%%%%%%%%%%%%%%%%%%%%%%%%%%%%%%%%%%%%%%%%%%%%%%%%%%%%%%%%%
\noindent
%%
\section*{Slot 1: Techniques and Tools I}
Monday, June 6th, 14:00--15:30

\begin{enumerate}[resume]
  %% 
\item {\large\bf Transfer of methods from modern theoretical physics to neuroscience}\\[1ex]
  %%(topics 1.1, 1.2, 1.3, 1.5)\\[1ex]
  {\bf Chair:} David (Moritz)\\[1ex]
  {\bf Focus of interest of:} David, Lukas, Dima, Tobias, Jannis\\[1ex]
  {\bf Proposed participants:} David, Claudia, Michael D, Tobias, Lukas, Hannah, Markus, Dima, Moritz\\[1ex]
%   \begin{itemize}
%   \item path-integral formulation for measure like rate and correlations and their distributions
%   \item renormalization methods
%   \item finding Hamiltonian-like object describing time evolution of neural networks
%   \item unified mean-field framework
%   \item possibly make reference to topic 1.4 (Neural-field models)
%   \end{itemize}
  %%
  {\bf Schedule:}\\[1ex]
  We divide the session into 3 parts, the first discussing
  past/published projects, the second ongoing projects and the third
  one future projects. The format will be that each session will have
  one person who presents a collection of slides (2-3 slides per
  project) on all corresponding projects in that category
  (past,ongoing,future). The presentation should not take longer than
  15 minutes. Afterwards we will have 5 minutes for immediate
  questions. At the end of the session, we will then have 30 minutes
  for in-depth discussion.\\[1ex]
  Slides:\\[1ex]
  Please prepare 2-3 slides on each of your projects with the main
  idea, relation to experiments and a sketch of the methods/tools
  involved. Unfortunately, there won't be time to go into depth for
  the methods. We should rather get an overview which methods are/were
  already employed at our institute in order to see where we can find
  cross-links or further methods for future projects. In addition to
  the content of future projects, we will also discuss their
  organization.
  %% 
\item {\large\bf Bayesian statistics}\\[1ex]
  %%(topic 1.10)\\[1ex]
  {\bf Chair:} Luca (Emiliano)\\[1ex]
  {\bf Focus of interest of:} Vahid, Luca, Nicole, Emiliano, Alper, Sven, Junji\\[1ex]
  {\bf Proposed participants:}Pietro, Daniel B, Sepehr, Sacha, Julia, Jakob, Junji, Nicole, Robin, Luca, Jenia, Sven, Emiliano, Jyotika, Tammo\\[1ex]
  {\bf Schedule:}
  %%
  \begin{itemize}
  \item Bayesian-logic tutorial: The Monty Hall problem (15--20~min)
  \item discussion/questions on the main points of Bayesian logic (15--20~min)
  \item brief presentation of an open inference problem from neuroscience by
    Sacha (5--10~min)
  \item hands-on: group work and discussion (possibly dividing into small
    groups) on the Bayesian analysis and setup of the presented
    neuroscientific inference problem (according to the guidelines of the
    tutorial) (remaining time, c.~40~min)
  \item round-up and literature suggestions (last 5--10~min)
  \end{itemize}
  % \begin{itemize}
  % \item Bayesian approach to estimating firing rate of a single neuron in a single trial 
  % \item biological and physical motivation 
  % \item Bayesian modeling in behavioral neuroscience
  % \item hidden patterns in big data
  % \item statistics beyond p-values
  % \item (coordinate with session 18)
  % \end{itemize}
  %% 
\item {\large\bf Machine learning and data mining}\\[1ex]
  %%(topic 1.12)\\[1ex]
  {\bf Chair:} Vahid (Alper)\\[1ex]
  {\bf Focus of interest of:} Vahid, Alper, Abigail, Jeyathevy, Robin\\[1ex]
  {\bf Proposed participants:} Tom, Alper, Simone, Renato, Daniel M, Steffen, Jeyashree, Michael B, Max, Philipp, Johanna, Rembrandt, Abigail, Fred\\[1ex]
  %%
  {\bf Schedule:}
  \begin{itemize}
  \item short introduction into Bayesian statistics (Luca/Max; up to 15 mins)
  \item interfacing ML and Bayes (Alper, Sepehr; 15mins)
    \begin{itemize}
    \item (Bayesian) Linear Regression
    \item MLE/MAP 
    \end{itemize}
  \item state-space analysis (Renato, 15mins) 
  \item clustering analysis (Alper, Daniel, 15mins)
    \begin{itemize}
    \item k-means
    \item DBSCAN
    \item Expectation Maximization
    \end{itemize}
  \item open discussion, e.g.~how to use ML on our problems or in general, problems and solutions regarding our work, in depth discussion
  \end{itemize}
  % \begin{itemize}
  % \item applying ML to neuronal data and network analysis
  % \item analysis of parallel spike trains 
  % \item clustering algorithms 
  % \end{itemize}
  %% 
\item {\large\bf Interpretation of spike rates}\\[1ex]
  %%(topic 1.7)\\[1ex]
  {\bf Chair:} Robin (Jakob)\\[1ex]
  {\bf Focus of interest of:} Robin, Vahid, Luca, Jakob\\[1ex]
  {\bf Proposed participants:} Inga, Espen, Jannis, Sonja, Carlos, Marcel, Lyuba, Jeyathevy\\[1ex]
  %% 
  {\bf Schedule:}\\[1ex]
  Paper presentations (10-15min each):
  \begin{itemize}
  \item Barlow H. The neuron doctrine in perception. In Gazzaniga M, editor, The Cognitive
    Neurosciences. Boston: MIT Press, 1994
    (Jakob)
  \item Quiroga, R. Q., Reddy, L., Kreiman, G., Koch, C., \& Fried, I. (2005). Invariant visual
    representation by single neurons in the human brain. Nature, 435(7045), 1102-1107 (Carlos)
  \item Romain Brette, (2015), Philosophy of the Spike: Rate-Based vs. Spike-Based Theories of the Brain (Robin)
  \end{itemize}
\end{enumerate}

\section*{Slot 2: Techniques and Tools II}
Tuesday, June 7th, 9:00--10:30

\begin{enumerate}[resume]
  %% 
\item {\large\bf Good programming practices, design of extendable software \& documentation}\\[1ex]
  %%(topics 2.1, 2.6)\\[1ex]
  {\bf Chair:} Julia (Steffen)\\[1ex]
  {\bf Focus of interest of:} Julia, Steffen\\[1ex]
  {\bf Proposed participants:} Daniel B, Sacha, Julia, Claudia, Jakob, Inga, Junji, Tobias, Steffen, Jeyashree, Jenia, Espen, Jannis, Max, Lukas, Markus, Abigail, Tammo\\[1ex]
  %% 
  {\bf Schedule:}
  \begin{itemize}
  \item Introduction to PEP8 standard [Julia, 15']
  \item Aspects of clean code (based on 'Clean Code: A Handbook of Agile Software Craftsmanship' by Robert C. Martin) [Julia, 15']
  \item Good Programming Practices - An Overview and Introduction in NEST [Tammo, 30']
  \item Code Quality Assurance and Problems in Elephant [Junji, 15']
  \item Discussion Time [everybody, 15']
  \end{itemize}
  %% 
\item {\large\bf How to use git and backup technologies}\\[1ex]
  %%(topic 2.8)\\[1ex]
  {\bf Chair:} Michael B (Jakob)\\[1ex]
  {\bf Proposed participants:} David, Tom, Simone, Renato, Nicole, Robin, Luca, Marcel, Dima\\[1ex]
  {\bf Schedule:} \alert{TBA}
  % \begin{itemize}
  % \item (tutorial?)
  % \end{itemize}
  %% 
\item {\large\bf Visualization techniques, visual analytics}\\[1ex]
  %%(topic 2.10)\\[1ex]
  {\bf Chair:} Maximilian \\[1ex]
  {\bf Proposed participants:} Alper, Sepehr, Daniel M, Michael D, Sven, Jyotika, Philipp, Johanna, Hannah, Moritz\\[1ex]
  {\bf Schedule:} 
  \begin{itemize}
  \item VisNEST [Maximilian, 10']
  \item Correlation Analyzer [Pietro, 10']
  \item VIOLA (4x4 Visualizer) [Johanna, 10']
  \item Visiphant and NEST-Elephant Multiview [Michael, 10']
  \item Generation and visualization of graphs with python-igraph [Maximilian, 10']
  \item pyQT [Alper, 10']
  \item maya-vi [Michael]
  \item Discussion
    \begin{itemize}
    \item Scripts and command-line tools vs. GUI
    \item Bring your plots - favorite or problematic plots
    \end{itemize}
  \end{itemize}
  Notes:
  \begin{itemize}
  \item see books recommended by Benni W
  \end{itemize}
  %% 
\item {\large\bf Research-data management}\\[1ex]
  %%(topics 2.5, 2.7)\\[1ex]
  {\bf Chair:} Carlos (Lyuba)\\[1ex]
  {\bf Focus of interest of:} Carlos, Lyuba, Frederic\\[1ex]
  {\bf Proposed participants:} Pietro, Michael B, Emiliano, Sonja, Carlos, Rembrandt, Fred, Lyuba, Jeyathevy\\[1ex]
  {\bf Schedule:} \alert{TBA}\\[1ex]
  Notes:
  \begin{itemize}
  \item data sharing and access to databases (e.g.~rodent-brain data through web clients)
  \item metadata annotation
  \item digitization of the research process (workflows)
  \end{itemize}
  %% 
\end{enumerate}

\section*{Slot 3: Data-analysis applications}
Tuesday, June 7th, 11:00--12:30

\begin{enumerate}[resume]
  %% 
\item {\large\bf Spatio-temporal patterns}\\[1ex]
  %%(topic 3.5)\\[1ex]
  {\bf Chair:} Michael D (Emiliano)\\[1ex]
  {\bf Focus of interest of:} Emiliano, Carlos, Michael D, Alper, Pietro, Vahid, (Max), (Karolina)\\[1ex]
  {\bf Proposed participants:} Pietro, Sepehr, Renato, Daniel M, Jakob, Junji, Luca, Jenia, Espen, Philipp, Johanna, Marcel, Lyuba, Nicole, Jeyathevy\\[1ex]
  {\bf Schedule:} \alert{TBA}\\[1ex]
  Notes:
  \begin{itemize}
  \item correlation structure in artificial neural networks
  \item analyzing spiking activity in multi-area model
  \item detection of spatio-temporal patterens in deep networks
  \item correlation structure in macaque motor cortex
  \end{itemize}
  %% 
\item {\large\bf Graph theory: applications and toolboxes}\\[1ex]
  %%(topics 2.9, 3.3)\\[1ex]
  {\bf Chair:} Abigail (Jyotika)\\[1ex]
  {\bf Focus of interest of:} Abigail, Robin, Jyotika, Jeyashree, Max\\[1ex]
  {\bf Proposed participants:} Daniel B, David, Alper, Simone, Claudia, Michael D, Tobias, Steffen, Jeyashree, Sven, Max, Jyotika, Sonja, Abigail, Dima, Moritz, Tammo\\[1ex]
  {\bf Schedule:} \alert{TBA}\\[1ex]
  Notes:
  \begin{itemize}
  \item graph-theoretical analysis of biological neural networks 
  \item graph-theoretical analysis of artificial neural networks
  \item graph-theoretical analysis of neural data
  \item How can anatomical or functional data benefit from graph-theoretical analysis?
  \end{itemize}
  %% 
\item {\large\bf Brain maps and connectomics}\\[1ex]
  %%(topic 3.4)\\[1ex]
  {\bf Chair:} Rembrandt (Sacha)\\[1ex]
  {\bf Focus of interest of:} Rembrandt, Sacha, (Renato)\\[1ex]
  {\bf Proposed participants:} Sacha, Julia, Lukas, Hannah, Rembrandt, Markus, Fred, (Nicole)\\[1ex]
  {\bf Schedule:} \alert{TBA}\\[1ex]
  Notes:
  \begin{itemize}
  \item folding patterns of ferret cortex in developing brains
  \item features of new connectivity matrix of macaque visual cortex
  \end{itemize}
  %% 
\item {\large\bf Applying our insights and methods to networks/data outside neuroscience (knowledge export)}\\[1ex]
  %%(topic 1.8)\\[1ex]
  {\bf Chair:} Johanna (Max)\\[1ex]
  {\bf Focus of interest of:} Johanna, Max\\[1ex]
  {\bf Proposed participants:} Inga, Robin, Michael B, Emiliano, Jannis, Carlos, Tom, Jakob, Luca, Jeyashree\\[1ex]
  	{\bf Schedule:}\\[1ex]
	The idea is to study concrete examples where techniques from our
	work can be applied to other fields (and what we can learn from other
	related fields). The main focus should be on networks of spiking
	neurons, i.e., how to construct and how to analyze them.  The
	session will be composed of short presentations of examples 		from literature/experience, discussions and the development of concrete ideas.\\	
  	\begin{itemize}
  	\item Introduction
  	\item Tom on Chapuis2014 "The variability of tidewater-glacier calving: origin of event-size and interval distributions"
  	\item Carlos on Cheng2014 "Can cascades be predicted?" and on Cheng2016 "Do cascades recur?"
  	\item Max on time series forecasting (tba)
  	\item Johanna on Lymperopoulos2015 "Online social contagion modeling through the dynamics of Integrate-and-Fire neurons"
  	\item Brainstorming/Collection and discussion of other ideas (e.g., Max on train system, finance, weather)
  	\item Wrap-up/Conclusions
  \end{itemize}
\end{enumerate}

\section*{Slot 4: Dynamics of neural systems}
Wednesday, June 8th, 9:00--10:30

\begin{enumerate}[resume]
  %% 
\item {\large\bf Dynamical-system analysis of neural networks}\\[1ex]
  %%(topic 4.2)\\[1ex]
  {\bf Chair:} Jannis (Sven)\\[1ex]
  {\bf Focus of interest of:} Sven, Jannis,  Tom\\[1ex]
  {\bf Proposed participants:} David, Tom, Alper, Claudia, Michael D, Inga, Junji, Jeyashree, Sven, Emiliano, Max, Jyotika, Philipp, Lukas, Moritz\\[1ex]
  %%
  {\bf Schedule:}
  \begin{itemize}
  \item Short introduction to Lyapunov exponents (characterizing
    chaos) in spiking systems, for example based \url{http://wwwold.fi.isc.cnr.it/users/alessandro.torcini/ARTICOLI/agt_pre2015.pdf}
  \item The participants form small groups to try out the presented
    ideas directly in spiking simulations with NEST.
  \end{itemize}
 % \begin{itemize}
 %  \item chaos and instability 
 %  \item characterization of dynamical systems (degrees of freedom, attractor dimension, max Lyapunov exponents)
 %  \item relation between breakdown of linear models and phase transition in networks of non-linear (analog or spiking) neurons
 %  \end{itemize}
  %% 
\item {\large\bf Synaptic and structural plasticity and homeostasis}\\[1ex]
  %%(topic 4.3)\\[1ex]
  {\bf Chair:} Renato (Phillip)\\[1ex]
  {\bf Focus of interest of:} Phillip, Renato, Claudia, Sepehr, Dima\\[1ex]
  {\bf Proposed participants:}Pietro, Daniel B, Sepehr, Robin, Jenia, Jannis, Abigail, Dima\\[1ex]
  {\bf Schedule:} \alert{TBA}\\[1ex]
  Notes:
  \begin{itemize}
  \item features of synapses that allow system to learn and adapt (incl.~homeostasis) 
  \item interaction of plasticity and activity on different time scales 
  \item homeostasis and robustness of networks
  \item similar output despite different system properties
  \end{itemize}
  %% 
\item {\large\bf Brain-activity features: functional relevance or byproduct?}\\[1ex]
  %%(topic 4.8)\\[1ex]
  {\bf Chair:} Junji (Hannah)\\[1ex]
  {\bf Focus of interest of:} Hannah, Junji\\[1ex]
  {\bf Proposed participants:} Julia, Jakob, Steffen, Hannah, Rembrandt, Fred, Marcel, Lyuba\\[1ex]
  %%
  {\bf Schedule:}
  \begin{itemize}
  \item Interactive perspective presentation "Relations between neuronal noise and signal (or ongoing and response activities)" by Junji (45 min)
  \item Interactive perspective presentation "Oscillation as filtered noise" by Hannah (30 min)
  \item Open discussion (15 min)
  \end{itemize}
% \begin{itemize}
%   \item oscillations 
%   \item spontaneous activity=noise?
%   \item In what state do neural circuits operate? 
%   \item Which description level suffices to describe input?
%   \end{itemize}
  %% 
\item {\large\bf Forward models \& biophysically realistic multi-scale activity}\\[1ex]
  %%(topics 4.5, 4.7)\\[1ex]
  {\bf Chair:} Espen (Daniel M)\\[1ex]
  {\bf Focus of interest of:} Espen, Daniel M\\[1ex]
  {\bf Proposed participants:} Simone, Sacha, Renato, Daniel M, Tobias, Luca, Espen, Michael B, Sonja, Carlos, Johanna, Markus, Tammo, Nicole, Jeyathevy\\[1ex]
  {\bf Schedule:}
  %%
  \begin{itemize}
  \item 00-10 min: Espen: Introduction, Forward modeling of extracellular potentials
  \item 10-20 min: Sonja: spike-LFP and spike synchrony - LFP relation
  \item 20-30 min: Carlos: Interpretation and biophysics underlying the fMRI signal (dead salmon fMRI controversy?) 
  \item 30-40 min: Markus: Mass signal-prediction in macroscopic neuronal network models
  \item 40-50 min: Sacha: Why do certain features in electrophysiological data emerge on a specific scale (e.g., alpha-band activity in EEG spectra)?
  \item 50-60 min: Daniel: Origin of low frequencies in multi-area models
  \item 60-70 min: Espen: Forward modeling schemes for ECoG, EEG and MEG signals
  \item 70-90 min: Discussion + misc.
  \end{itemize}
% \begin{itemize}
%   \item LFP modeling 
%   \item modeling what we measure; bridging gap between simulations and measured signals like EEG, MEG, and fMRI
%   \item understanding experimental setups 
%   \item fMRI, optogenetics, MEG -- Why relevant?
%   \item How to incorporate knowledge from receptor mapping into multi-area models?
%   \item multi-scale modeling incorporating inter-laminar and inter-areal activity 
%   \item moving from random networks to realistic cortical microcircuits 
%   \item Can we design stimuli that reveal the underlying circuit?
%   \end{itemize}
  %% 
\end{enumerate}

\section*{Slot 5: Function of neural systems}
Wednesday, June 8th, 11:00--12:30

\begin{enumerate}[resume]
  %% 
\item {\large\bf Coding and information transfer in biological neural networks}\\[1ex]
  %%(topic 5.2)\\[1ex]
  {\bf Chair:} Sepehr (Daniel M)\\[1ex]
  {\bf Focus of interest of:} Sepehr, Daniel M, Jeyathevy\\[1ex]
  {\bf Proposed participants:} Pietro, David, Sacha, Julia, Renato, Daniel M, Michael D, Tobias, Robin, Luca, Espen, Sven, Emiliano, Jyotika, Philipp, Sonja, Carlos, Lukas, Hannah, Rembrandt, Marcel, Lyuba, Jeyathevy\\[1ex]
  {\bf Schedule:} 
  \begin{itemize}
  \item Background and overview (10 min)
  \item Mutual info, Granger Causality, and transfer entropy and their shortcomings (10 min)
  \item New metrics (10 mins)
  \item When to use which metric (5 min)
  \item Discussion (10 mins)
  \item Break (10 mins)
  \item Summary  (5 mins)
  \item Hands-on session (30 mins)
  \end{itemize}
  % Notes:
  % \begin{itemize}
  % \item comparability and interpretability of methods like Granger
  %   causality, DCM, transfer entropy 
  % \item How is information encoded in spikes? Are spikes the only
  %   carrier? Does it make sense to discuss a ``neural code'' at all?
  % \end{itemize}
  %% 
\item {\large\bf Bayesian computing in biological neural networks}\\[1ex]
  %%(topic 5.9)\\[1ex]
  {\bf Chair:} Jakob (Luca)\\[1ex]
  {\bf Focus of interest of:} Jakob, Luca, Daniel M, Jenia, Tom\\[1ex]
  {\bf Proposed participants:} Daniel B, Jakob, Inga, Junji, Nicole, Max, Moritz\\[1ex]
  %%
  {\bf Schedule:}
  \begin{itemize}
  \item introduction (20min, Jakob)
  \item hands on: Boltzmann learning; adapting internal representation to match target distribution; probabilistic inference (50min; Vahid, Jakob)
  \item discussion (20min)
  \end{itemize}
  %% 
\item {\large\bf Measures of computational capability of neural networks}\\[1ex]
  %%(topic 5.8)\\[1ex]
  {\bf Chair:} Tom (Sven)\\[1ex]
  {\bf Focus of interest of:} Tom, Sven, Daniel B\\[1ex]
  {\bf Proposed participants:} Tom, Simone, Claudia, Jannis, Johanna, Markus, Dima\\[1ex]
  %% 
  {\bf Schedule:}
  \begin{itemize}
  \item Types of computation (tasks) (Tom)
    \begin{itemize}
    \item brief overview of different types of tasks
      (e.g. memory, classification, inference, sequence generation, prediction of dynamical systems, \ldots)
    \item examples from neuroscience-literature where these tasks are implemented by neural networks
    \item measures of task complexity (e.g.~dimensionality)
    \end{itemize}
  \item Discussion:
    What tasks can you think of? What tasks does the (biological) brain perform?
    Can we classify tasks?
  \item Computational capability of neural networks\\[1ex]
    Which measure are used/common in neuro-literature (short presentations by session participants, 1-2 slides/measure)?
  \item Discussion
    \begin{itemize}
    \item How do the individual measures relate to specific tasks? What measure is reasonable for what task?
    \item Which measures are task specific? Which measures are not (or less) task specific?
    \item What aspects of the task do the different measures account for? 
    \end{itemize}
  \end{itemize}
 % \begin{itemize}
 %  \item standardized benchmarks and metrics for performance, memory, transfer of information, \ldots
 %  \end{itemize}
  %% 
\item {\large\bf Deep learning}\\[1ex]
  %%(topic 5.12)\\[1ex]
  {\bf Chair:} Jenia (Jakob)\\[1ex]
  {\bf Focus of interest of:} Jenia, Jakob\\[1ex]
  {\bf Proposed participants:} Alper, Steffen, Jeyashree, Jenia, Michael B, Abigail, Fred, Tammo\\[1ex]
  {\bf Schedule:} \alert{TBA}\\[1ex]
  Notes:
  \begin{itemize}
  \item multi-modal learning in very deep recurrent neural networks
  \item deep denoising autoencoders
  \end{itemize}
  %% 
\end{enumerate}

%%%%%%%%%%%%%%%%%%%%%%%%%%%%%%%%%%%%%%%%%%%%%%%%%%%%%%%%%%%%%%%%%%%%%%%%

\end{document}

%%% Local Variables:
%%% mode: latex
%%% TeX-master: t
%%% End:
